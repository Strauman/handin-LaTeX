%§ex=Making exercises
%§fp=Page formatting commands
%§ref=General reference
%!TEX root = main.tex
%!TEX root = oblig1.tex
\usepackage[utf8]{inputenc}
\usepackage{graphicx}
\usepackage{lastpage}
\usepackage{scrextend}

%!TEX root = oblig1.tex
%\BDOC
%§ex
% Here are commands related to creating exercises
%\EDOC
\makeatletter
%Macro containing exercise number
\def\exerciseNr{0}
% \BDOC
% \problem{text}
% This command will print out a problem header. For example \problem{1}
% prints a nice big header \!textbf{Problem 1}
% \EDOC

\newcommand{\problem}[1]{
% Update exercise number
  \def\exerciseNr{#1}
% Move margins
  \begin{addmargin}{-1.5em}
% Problem and problem number
    {\normalfont\Large\bfseries \@tr{Problem} #1}
% Move margins back
  \end{addmargin}
}
% \BDOC
% \pproblem{text}
% This command will print out a part problem header based on what problem you are on.
% For example if you already have done \problem{1}, then \pproblem{a}
% prints a nice big header \!textbf{(1a)}
% \EDOC
\newcommand{\pproblem}[1]{
% Make some space above
  \vspace*{1em}
% Move margins
  \begin{addmargin}{-0.5em}
% Write out problem number and letter
    {\normalfont\Large\bfseries \exerciseNr #1)}
% Move margins back
  \end{addmargin}
% Make some space below
  \vspace*{1em}
}
\makeatother
 \usepackage{fancyhdr}
\usepackage{geometry}
% Mattematikk
\usepackage{amsmath}
\usepackage{mathtools}
\usepackage{bm}
\usepackage{esint}
\usepackage{iflang}
 %!TEX root = ../main.tex
\makeatletter
\def\@tr#1{
  \ifcsname tr@#1\endcsname%
    \csname tr@#1\endcsname%
  \else%
    #1%
  \fi%
}
% \set@tr{english}{foreign}
\newcommand{\set@tr}[2]{
  \expandafter\def\csname tr@#1\endcsname{#2}
}
%!TEX root = ../main.tex
\set@tr{containspages}{Contains \pageref{LastPage} pages, front page included}
\set@tr{pagetext}{Page \thepage~of \pageref{LastPage}}
 \IfLanguageName{norsk}{%!TEX root = ../main.tex
\makeatletter
\set@tr{Problem}{Oppgave}
\set@tr{problem}{oppgave}
\set@tr{pagetext}{Side \thepage~av \pageref{LastPage}}
\set@tr{containspages}{Inneholder \pageref{LastPage} sider, inkludert forside.}
\makeatother
 }{}
\makeatother
 \newif\ifshowoff
\showofffalse
%!TEX root = main.tex
% \ProvidesPackage{settable}
%   [2018/03/17 v0.01 Package for creating settable macros with default values.]

%%% Usage:
%%% \settable{hello}
%%% if now \@hello is called, a warning is displayed and "\hello not set" is printed as warning
%%% \hello{world}
%%% if now \@hello is called, it prints "world"
%%% \@hello@noerror gives the returning content and empty without error if no content set.
%%% \ifset@hello{true}{false}
% \BDOC
% §ref
% \settable{text}
% The text you enter would be a macro. See example:
% \EDOC
%\BEX
% \settable{hello}
% %if now \@hello is called,
% % a warning is displayed with
% % the text "\hello not set"
% \hello{world}
% % if now \@hello is called, it prints "world"
% \@hello@noerror gives the returning
% content and empty without error if no content set.
% \ifset@hello{true}{false}
%\EEX
\makeatletter
\let\ea = \expandafter
\newcommand{\settable}[2][\@nil]{
  %%% example call \settable[\actionIfNotSet]{test}

  %% below equiv to
  %% \def\test##1{
  %%     \def\@test{##1}
  %%     \def\@test@noerror{##1}
  %% }
  \ea\def\csname #2\ea\endcsname##1{
    \ea\def\csname @#2\endcsname{##1}
    \ea\def\csname @#2@noerror\endcsname{##1}
    \ea\def\csname isset@#2\endcsname{1}
  }
  \ea\def\csname ifset@#2\endcsname##1##2{
    \ifcsname isset@#2\endcsname%
    ##1
    \else
    ##2
    \fi
  }

  \ea\def\csname default@#2\endcsname{#1}%
  \ea\def\csname @#2@noerror\endcsname{#1}%

  \ea\ifx\csname default@#2\endcsname\@nnil
    \ea\def\csname default@#2\endcsname{
      \textbackslash #2%
      {\ea\@latex@warning{\ #2 not given}}
    }
  \fi

  \ea\def\csname @#2\endcsname{
    \csname default@#2\endcsname
  }
}

\ifshowoff
% \renewcommand{\settable}[2][\@nil]{
%   \ea\def\csname #2\ea\endcsname##1{Q}
%   \ea\def\csname @#2\endcsname{\textbackslash#2}
%   \ea\def\csname default@#2\endcsname{\textbackslash#2}
%   \ea\def\csname @#2@noerror\endcsname{\@empty}
%   \ea\def\csname ifset@#2\endcsname##1##2{##2}
% }
\@latex@warning{Settable: showing off, no settable is actually set!}
\renewcommand{\settable}[2][\@nil]{
  \ea\def\csname #2\endcsname##1{\relax}%
  \ea\def\csname @#2\endcsname{}%
  \ea\def\csname @#2@noerror\endcsname{}%
  \ea\def\csname ifset@#2\endcsname##1##2{##2}
  \ea\def\csname default@#2\endcsname{#1}%
  \ea\def\csname @#2@noerror\endcsname{#1}%
  \ea\def\csname @#2\endcsname{\textbackslash #2}
}
\fi
% \endinput
\makeatother
 %\BDOC
%§fp
% This package redefines \maketitle.
% Here are some front-page commands. See layout.pdf for where they will appear.
% These commands all have to be executed in the preamble (that is after \documentclass and before \begin{document})\\
% The \title and \author commands are as per usual:
%\EDOC
%\BDOC
%\title{title}
%\EDOC
\settable{title}
%\BDOC
%\author{your name}
%\EDOC
\settable{author}
%\BDOC
%\logo{path/to/image}
% If you want an image below the title, you provide the path to the image here
%\EDOC
\settable{logo}
%\BDOC
%\coursename{text}
\settable{coursename}
%\EDOC
%\BDOC
%\coursetitle{text}
% The front page will show coursename - coursetitle on a "subtitle" format
%\EDOC
\settable{coursetitle}
% \BDOC
% \institute{text}
% Shows as text on bottom
% \EDOC
\settable{institute}
% \BDOC
% \containspages{text}
%   Here you can set a string that shows on bottom. Default is\\
%   \containspages{Contains \pageref\brackets{LastPage} pages, front page included}
% \EDOC
\settable{containspages}
% \BDOC
% \pagetext{string}
% This is the text that is on the bottom right corner reading "Page x of y". Default is
% \pagetext{Page \thepage~of \pageref{LastPage}}
% \EDOC
\settable{pagetext}

\setlength\parindent{0pt}
% Smaller margins
\geometry{paper=a4paper, bottom=3cm, top=3cm, footnotesep=3cm}
% Header and footers
\fancyhf{}
% Make top header line wider
\addtolength\headwidth{4em}
\fancyheadoffset{2em}
\pagestyle{fancy}
 %!TEX root = main.tex
\makeatletter
\let\old@maketitle = \maketitle
\def\maketitle{
	\old@maketitle
	\thispagestyle{empty}
	\clearpage
}
\def\@maketitle{%
	%!TEX root = main.tex
\makeatletter

\begin{center}
  \ifshowoff
    \Huge{\textbackslash logo}\\
    \Large{\@title} \\[1.5cm]
  \else
  \ifset@logo{
    \includegraphics[scale=1]{\@logo}\\[0.5cm]
    \Large{\@title} \\[1.5cm]
  }{
    \ea\@latex@warning{Use \noexpand\logo{path/to/image} to set a logo on the front page}
    \Huge{\@title} \\[1.0cm]
  }
  \fi
\end{center}
\begin{center}
  \textbf{\@coursename - \@coursetitle}\\[1cm]
  \textbf{\@author} \\[1cm]
  \textbf{\today} \\[3cm]
\end{center}

\begin{center}
  \vfill
  \ifset@containspages{
    \@containspages
  }{
    \@tr{containspages}
  }\\[0.5cm]
  \textsc{\@institute}
\end{center}
\makeatother
 	\rhead{\@author}
	\lhead{\@coursename - \@coursetitle}
	\ifset@pagetext{
		\rfoot{\@pagetext}
	}{
		\rfoot{\@tr{pagetext}}
	}
}
\makeatother
 